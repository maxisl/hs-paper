\section{Study on Usage Patterns in LLM and Text-To-Image Generation Tool Interactions}
\label{sec:study-on-usage-patterns-in-llm-and-text-to-image-generation-tool-interactions}
The following section contains the study part of our research.
We first state the objective of our study and then explain the individual steps that will be
taken to obtain sufficient results.
%%%%%%%%%%%%%%%%%%%%%%%%%%%%%%%%% RESEARCH OBJECTIVE %%%%%%%%%%%%%%%%%%%%%%%%%%%%%%%%%
\subsection{Research Objective}
\label{subsec:research-objective}
% Intro and Research Objective of the study goal, the methodology, and the individual
% steps that will be taken
The main outlined goal of our research is to gain a fundamental understanding of user behavior in
conversation with Large Language Models and similar AI tools.
Our analysis aims to identify common patterns and strategies in those interactions.
Previous research indicates that users regularly face challenges and difficulties, especially
when trying to formulate effective prompts.
Through accumulation and analysis of qualified data samples, we aim to identify and understand these
challenges,
as well as investigate the impact and effect of user behavior on the effectiveness of model
responses.
Given the various kinds of available generative models, we want to examine differences in
prompting behavior according to model type as well.

Since related insights we have already highlighted suggest that reformulating search queries is a
popular strategy to improve results, we want to investigate if users apply this strategy in AI
conversations as well.
Furthermore, we want to assess the extent to which users show awareness of effective prompt
formulation strategies, such as few-shot learning, and whether they rely on appropriate language
that is machine and not human directed, thus showing comprehension that they are talking to an AI\@.

%%%%%%%%%%%%%%%%%%%%%%%%%%%%%%%%% RESEARCH METHOD %%%%%%%%%%%%%%%%%%%%%%%%%%%%%%%%%
\subsection{Research Method: ShareGPT and Midjourney}
\label{subsec:research-method:-sharegpt-and-midjourney}
% Information on the ShareGPT & Midjourney platforms, their user base, suitability for the
% study, and which data we are going to use
In order to obtain credible insights, we complement existing findings with real-world data.
Our study analyzes data samples from two different types of AI models.
First, we examine user interactions with ChatGPT, a generative NLP model, which has already been
described in more detail in Section~\ref{subsec:large-language-models-(llms)}.
These ChatGPT conversations were obtained from the website ShareGPT\footnote{\url{https://sharegpt.com/}}.
ShareGPT is an open platform, that allows its community to publicly share interactions they have
had with the ChatGPT model.
As of today, ShareGPT has accumulated nearly 300.000 saved user conversations.
Users can access shared conversations by following the corresponding link.
To gather 50 data samples, we have crawled the web in order to obtain enough links to conversations.
What makes ShareGPT particularly suitable for our use-case is the fact that the entire shared conversation
can be viewed by the observer as if they had personally conducted the interaction, allowing us to
gain a deeper understanding of the conversation dynamics and outcome.
It is to be noted however, that all publicly available conversations on ShareGPT were
intentionally shared.
Therefore, they may include a potential bias, as users presumably only share conversations which
they perceived as particularly funny, impressive, disturbing, or similar.
\newline

Midjourney in contrast, is a platform that focuses on AI-based image generation.

Users can interact with the model through Discord~\footnote{\url{https://discord.com/}} and submit
individual requests.
In order to generate an illustration, users have to enter a descriptive prompt, similar to
ChatGPT\@.
The description typically includes everything that should appear in the picture, but may also
encompass the desired mood, drawing style, or composition of the image being generated.
Notably, the Midjourney Bot does not understand grammar, sentence structure, or specific words like
humans do~\cite{midjourney_documentation_2023}.
Midjourney's developers actively encourage using fewer, but more precise and impactful words when
prompting the model.
For example, they suggest using ``gigantic'' instead of ``big'' in order to achieve
better results.
This recommendation stems from the fact that fewer words in a prompt intensify the influence each
individual word has on the final outcome.
However, it is important to mention that users have to strike a balance.
An adequate amount of precise words is mandatory, because anything that is not specified may be
randomized.
In addition to purely textual prompts, the platform allows image inputs as well.
Users may provide an image as a guideline or basis and include instructions about things to modify,
add, remove, or remodel.
The Midjourney platform on Discord has experienced rapid growth, and counts more than 17 million
members as of today.
Similarly to ShareGPT, we have accumulated 50 samples from real-world prompts of users and
corresponding results.
To do so, we accessed the Discord channel of Midjourney, where it is possible to observe all image
generations of other users including the initial textual inputs.
\newline

In order to verify observations and findings we have presented in Section~\ref{sec:background-and-related-work},
we examine exemplary real-world user interaction samples in the following.
By choosing 50 conversations from each ShareGPT's website as well as Midjourney's Discord
channel, we obtain a representative sample of average user behavior in both text and image targeting prompts.
We have defined dedicated categories according to which each sample is classified for both
ChatGPT and Midjourney.
For the language-focused ChatGPT conversations, the categories and specific sub-categories can be
seen in Table~\ref{tab:sharegpt-prompt-analysis-categories}.
Similarly, the categories and sub-categories for image-focused Midjourney prompts are listed
in Table~\ref{tab:midjourney-prompt-analysis-categories}.
Prompt modifiers have been the subject of recent related research~\cite{oppenlaender_taxonomy_2023}.
Generally speaking, modifiers act as a means of directing the output towards a desired outcome.
In our case, the various categories aimed to provide insights into the way users approach prompting
a model in general and how they may adapt their behavior during the course of a conversation in
order to tailor the output.
While existing taxonomies often classify individual parts of a prompt, and use
categories such as ``subject term'', ``style modifier'', or ``quality booster''~\cite{oppenlaender_taxonomy_2023}, we
investigate interactions as a whole.

To obtain a comprehensive and complete analysis, we followed a blueprint of separating the
categories into the main topics of motivation, shape of the prompt, and user behavior.
Motivation entailed the categories Type and Intent, shape of the prompt contained Length, Prompt
Setting, as well as Complexity, and user behavior was made up of Engagement, Refinement, and
Language.
Accordingly, in the first category we classified the prompt into common use case
subcategories, as in theory any kind of prompt is possible, because users are solely constrained to natural language
in any shape or form.
In the Intent category, it was mainly of interest what the user intended to achieve with their
individual prompts,
i.e.\ what they use the model for.
For the prompting setting of LLMs, we used subcategories based on related work in
Section~\ref{subsubsec:few-shot-learning-capabilities}.
We then tried to understand if for example longer, more complex queries with a
lot of modifiers improve the results or do not have the desired beneficial effect.
The category Engagement refers to the amount of exchange in the interaction.
If the user prompted the model multiple times (at least twice) during the course of the
conversation, we considered the interaction multi turn, otherwise single turn.
The prompt's complexity gave us an idea whether users leverage ChatGPT for simple
tasks, that they may otherwise quickly research using a search engine, or if they pose complex
questions that require expert-level knowledge.
The refinement degree of the prompt revealed if users were generally content with the initial
answer of the LLM, or if further elaboration was needed.
Finally, we differentiated use of formal and informal language.
In general, when a single conversation consisted of multiple prompts, we labeled it based on the
most frequently observed category or significant behavior.

We classified Midjourney interactions using a similar approach.
However since the motivation is generally obvious using an image creation AI tool, we emitted this
topic, and instead focused on the shape of the prompt and user behavior.
Shape of the prompt contained Input Type, Length, Composition, and Complexity,
and user behavior was made up of Refinement, Language, Clarity, and Satisfaction.
First of all, we differentiated between image- and purely language-based inputs.
We then considered the length of the prompt, and whether it consisted solely of keywords, one
or more sentences, or a mix of both.
In regard to length, we first examined the data samples, and came to the conclusion that a
grouping into short (1-3 words), medium (4-12), and long (12+) prompts made the most sense based
on the various lengths of prompts we have witnessed.
Next, we recorded the complexity of the whole prompt.
The Midjourney bot always generates four versions of the desired image.
It then allows users to either recreate variations or more detailed versions of one or more of
those four results.
Users can also re-execute the whole generation process.
We thus classified the prompt accordingly in the refinement category.
Finally, we recorded the clarity of the prompt and satisfaction levels based
on the observed user behavior.
If a user created variations or more detailed versions of the result, we assumed they were
generally satisfied.
Analogously, we assumed dissatisfaction if they regenerated the whole image.

% set linewidth 0,2281875 for \begin{tabular}{p{0.1\linewidth}p{0.81275\linewidth}}
\begin{table}[]
    \centering
    \caption{ShareGPT Prompt Analysis Categories}
    \begin{tabular}{@{}llll@{}}
        \toprule
        Type        & Intent        & Length              & Setting   \\ \midrule
        Question \hfill 36\%    & Information \hfill 36\%   & Long \hfill 28\%  & Zero
        Shot \hfill 90\% \\
        Statement \hfill 2\%    & Advice \hfill 10\%        & Med \hfill 34\% & One Shot \hfill 8\% \\
        Task-based \hfill 62\%  & Opinion \hfill 0\%       & Short \hfill 38\% & Few Shot \hfill
        2\% \\
                                & Suggestion \hfill 34\%   &                    &           \\
                                & Entertainm. \hfill 10\% &                    &           \\
                                &                           &                    &           \\
                                &                            &                    &           \\
        \toprule
        Engagement              & Complexity    & Refinement         & Language  \\ \midrule
        Single Turn \hfill 48\%  & Simple \hfill 48\%       & None     \hfill 66\%           &
        Formal \hfill 60\%\\
        Multi Turn \hfill 52\%   & Intermed. \hfill 38\% & Once     \hfill 22\%          &
        Informal \hfill 40\%\\
                                & Complex \hfill 14\%      & Mult. \hfill 12\%           &           \\ \\
        \bottomrule
    \end{tabular}
    \label{tab:sharegpt-prompt-analysis-categories}
\end{table}

\begin{table}[]
    \centering
    \caption{Midjourney Prompt Analysis Categories}
    \begin{tabular}{@{}llll@{}}
        \toprule
        Type                    & Length                & Composition           & Complexity   \\ \midrule
        Language \hfill 92\%    & Long \hfill 38\%      & Keywords \hfill 26\% & Simple  \hfill
        38\%     \\
        Image   \hfill 8\%     & Med \hfill 54\%        & Sentence \hfill 42\%      &
        Intermediate \hfill 40\% \\
                                & Short \hfill 8\%      &  Mix \hfill 32\%          & Complex \hfill 22\%     \\
        \\
        \toprule
        Refinement          & Language & Clarity & Satisfaction \\
        \midrule
        None \hfill 76\% & Formal \hfill 94\%       & Clear \hfill 92\%     & Satisfied \hfill 26\% \\
        Variation \hfill 10\% & Informal \hfill 6\% & Ambiguous  \hfill 8\% & Dissatisfied \hfill 16\% \\
        Regen. \hfill 14\% & & & Unclear \hfill 58\% \\\\
        \bottomrule
    \end{tabular}
    \label{tab:midjourney-prompt-analysis-categories}
\end{table}

%%%%%%%%%%%%%%%%%%%%%%%%%%%%%%%%% STUDY RESULTS %%%%%%%%%%%%%%%%%%%%%%%%%%%%%%%%%
\section{Study Results}
\label{sec:study-results}
This section contains the analysis of our study results.
We provide detailed numbers and offer insights based on the collected data.
%%%%%%%%%%%%%%%%%%%%%%%%%%%%%%%%% FINDINGS AND OBSERVABLE TRENDS %%%%%%%%%%%%%%%%%%%%%%%%%%%%%%%%%
\subsection{Findings and Observable Trends}
\label{subsec:findings-and-observable-trends}
% Listing of the results of the study, potentially segregated into categories that can be defined
% in advance
In the following section, we break down the sample analysis results by category.
The distribution of the sub-categories in the individual categories Prompt Type, Prompt
Intent, Prompt Length, and Prompt Setting can be seen in Figure~\ref{fig:chatgpt-categories-1},
whereas Figure~\ref{fig:chatgpt-categories-2} shows the numbers for the categories Engagement,
Complexity, Refinement, and Language.
In regard to prompt type, it became clear that the majority of users (54\%) use ChatGPT for task -based prompts, followed by questions (36\%), commands (8\%), and statements (2\%).
The most observed intents behind prompts were information gain (42\%) and asking for suggestions (34\%), followed by entertainment (10\%) and advice (10\%).
Only few users were asking for clarification on a subject matter (4\%).
We did not observe any prompts where users actively asked the chatbot for its
opinion (0\%), which we initially had estimated as an at least fairly common use case.
Prompt length was very evenly distributed, and we could not make out a clear preference of users.
Short (38\%), medium length (34\%), and long (28\%) prompts made up about a third of our samples
each.
We could clearly see the most often used prompting setting, however.
The vast majority of users relied on a zero shot approach (90\%), whereas only 8\% used a one shot,
and a mere 2\% a few shot setting.
Engagement in interactions was evenly distributed between multi turn (52\%) and single turn (48\%)
conversations, meaning that almost half of the observed chats ended after the initial answer of
the LLM\@.
Most prompts were of a simple nature (48\%), and slightly more than a third (38\%) could be
classified as intermediate, which left only 14\% as complex.
ShareGPT users only rarely refined their prompts multiple times (12\%) or once (22\%),
leaving a two thirds majority (66\%) of never refined queries.
Finally, we could observe a tendency towards formal language (60\%), which was used more often
than informal language (40\%).

Similarly, we analyzed the Midjourney data samples according to the predefined categories.
The corresponding data and distribution for the categories Prompt Type, Length, Composition, and
Complexity can be seen in Figure~\ref{fig:midjourney-categories-1}, for the categories Refinement, Language, Clarity, and Satisfaction in Figure~\ref{fig:midjourney-categories-2}.
We already explained that Midjourney allows users to also prompt with an existing image as part of
the input.
However, only very few (8\%) users have made use of this feature in our data sample.
The majority (92\%) relied on purely textual prompts.
Most queries were at least of medium length (54\%), or even long (38\%), and only 8\%
were classified as short.
The composition of the individual prompts was well-balanced between sentences (42\%), only keywords
(26\%), or a mix of both (32\%).
The same applies for the complexity.
Most prompts were on an intermediate level (40\%), closely followed by simple (38\%), and finally
complex prompts with a share of only 22\%.
Similarly to our ChatGPT samples, we observed only few refinements of Midjourney prompts.
More detailed regenerations of images made up 14\%, variations 10\%, and the rest (76\%) was not
refined
at all.
A clear distribution could be seen in regard to formality of language.
Users relied on formal language in almost all cases (94\%), and only very rarely on more informally
phrased prompts (6\%).
We identified 92\% of all prompts as clear in their intention, which left only 8\% as ambiguous.
Satisfaction of users was unfortunately often unclear (58\%), due to no apparent reactions of the
users to the final image.
However, for almost half of the samples we could either identify signs of satisfaction (26\%) or
dissatisfaction (16\%).

% TODO make an own subsection on difficulties of ordinary users in understanding how LLMs interpret
% natural language? in this subsec, we could focus on the example also displayed on midjourneys
% website: " If you ask for a party with “no cake,” your image will probably include a cake"
