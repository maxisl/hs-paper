%%%%%%%%%%%%%%%%%%%%%%%%%%%%%%%%% Introduction %%%%%%%%%%%%%%%%%%%%%%%%%%%%%%%%%

\section{Introduction}
\label{sec:introduction}
%% refer to https://intra.ece.ucr.edu/~rlake/Whitesides_writing_res_paper.pdf for tips on introduction?
%% INTRO - TOPIC

%One of the main challenges is
%% connectivity is a given
%Today,  is ever prevalent, which is reflected in the fact that
%
%
%% LLMs
%

%% Prompting


%
%% User interaction / input + what will be done?
%
%In general, the Introduction should have these elements:
%- The justification for these objectives / motivation:
%Why is the work important?
\sloppy % use sloppy to improve linebreaks - longer words do not overflow
Artificial Intelligence (AI) -based tools continually gain prominence as regularly leveraged tools in the
daily lives of millions of people.
In addition to typical AI applications such as recommendation systems or autonomous agents, generative
models are notably gaining popularity as well.
One of the most widely used implementations are Large Language Models (LLMs).
Since their main application is text generation in the form of an AI chatbot that can answer seemingly
any question a user might pose, an investigation of such interactions is particularly of interest.
Giving an LLM a task or asking it a question is commonly referred to as \("\)prompting\("\) the model.

%- The objectives of the work.
In this paper, we are going to investigate general user behavior that can be observed when interacting
with LLMs.
This investigation will provide an improved understanding of existing challenges users face when
using such models, as well as highlight optimization potential in order to enhance the output from
the model.

%- Background:
    %Who else has done what? How? What have we done previously?
Plenty of research has been conducted in the field of user interactions with LLMs.
% TODO list related work

%- Guidance to the reader:
%What should the reader watch for in the paper?
%What are the interesting high points? What strategy did we use?
Since the main part of this work will contain an analysis of real world examples after we have
sufficiently described all necessary concepts and prerequisites, the reader can expect to gain a better
understanding of the state of the art in user - LLM interactions.
To do so, we will leverage input data mainly gained from the website ShareGPT, % TODO cite
which stores voluntarily shared interactions of users with the most prominent LLM at the moment,
ChatGPT. % TODO cite

%- Summary/conclusion:
    %What should the reader expect as conclusion?
As a conclusion, we will summarize our findings and explain how and in which way we could recognize
findings from related research in our own data sample.