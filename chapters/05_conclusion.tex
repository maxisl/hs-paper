%%%%%%%%%%%%%%%%%%%%%%%%%%%%%%%%% CONCLUSION %%%%%%%%%%%%%%%%%%%%%%%%%%%%%%%%%
\section{Conclusion}
\label{sec:conclusion}

Throughout this paper we have explored user behavior in interactions with LLMs across multiple
dimensions.
The goal was to find prevalent human tendencies, understand existing habits, and identify recurrent
patterns.

To do so, we first explained the ever-growing importance of the subject, given the increasing
use of generative AI across all domains.
We then laid out fundamental concepts and explained existing findings from related research.
These concepts and findings built the foundation for a synthesis of our own real-world
data analysis with existing research in order to verify findings.
To obtain a comprehensive overview, we explained different kinds of LLMs and their
characteristics, highlighting ChatGPT and Midjourney in particular.
An extensive study of data samples that encompassed a categorization of each individual
entry according to eight predefined categories helped us to understand the current state of
human - LLM interactions.
We could observe that many users are still subject to biases and misunderstandings that make
effective prompting difficult.
During the discussion of our results, we reasoned why certain user dynamics exists, and which
actions LLM providers and developers as well as users themselves could undertake in order to
address prevalent issues.

Deeper exploration of this topic with a larger data sample is needed in order to
verify findings and gain a deeper understanding of existing interaction dynamics and is subject to
further research.
Overall, it is to be acknowledged that user prompting behavior is one of the most relevant
research topics in human centered computing and will gain even more importance in the near future.