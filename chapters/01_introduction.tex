%%%%%%%%%%%%%%%%%%%%%%%%%%%%%%%%% Introduction %%%%%%%%%%%%%%%%%%%%%%%%%%%%%%%%%

\section{Introduction}
\label{sec:introduction}
%% refer to https://intra.ece.ucr.edu/~rlake/Whitesides_writing_res_paper.pdf for tips on introduction?
%% INTRO - TOPIC

%Why is the work important?
\sloppy % use sloppy to improve linebreaks - longer words do not overflow
Artificial Intelligence (AI) -based tools continually gain prominence as regularly leveraged tools in the
daily lives of millions of people.
In addition to typical AI applications such as recommendation systems or autonomous agents, generative
models are notably increasing in popularity as well.
One of the most widely used implementations of generative models are Large Language Models (LLMs),
such as OpenAI's ChatGPT for example. % TODO cite
Such models mainly come in the form of text generating chatbots that can answer seemingly any question
a user might pose.
Nevertheless, it is challenging to optimize the output of the model, since it can vary depending
on the user input. % TODO cite?
Any request to an LLM, whether it is in the form of a task or a question, is commonly referred to as
\("\)prompting\("\)the model.
Due to the vast application possibilities and promising future developments of LLMs,
an exploration of user prompting behavior in interactions with these models is of particular interest.

%- The objectives of the work.
In this paper, we are going to explain the workings of LLMs and prompting, describe related research
in the realm of user - LLM exchange, and perform our own investigation of user behavior in these
interactions.
This investigation will provide an improved understanding of existing challenges users face when
dealing with such models, as well as highlight optimization potential in order to enhance generated
output.

%- Background:
    % Who else has done what? How? What have we done previously?
Plenty of research has been conducted in the field of user interactions with LLMs.
% TODO list related work + describe briefly

%- Guidance to the reader:
% What should the reader watch for in the paper?
% What are the interesting high points? What strategy did we use?
Since the main part of this work will contain an analysis of real world examples, the reader can
expect to gain a better understanding of user prompting behavior.
To obtain these insights, we will leverage input data mainly gained from the website ShareGPT, % TODO cite
which stores voluntarily shared interactions of users with the most prominent LLM at the moment,
ChatGPT. % TODO cite

%- Summary/conclusion:
% What should the reader expect as conclusion?
In the concluding section of this paper we will summarize our findings and explain how and in which way
we can recognize findings from related research in our own data sample.