\usepackage{enumitem}
% SELBST: tikz for drawing
\usepackage{tikz}
\usepackage{tabularx}
\usepackage{multirow}
\usepackage{pgfplots}
\pgfplotsset{compat=1.17}
\usepackage{pgf-pie}


%% The first command in your LaTeX source must be the \documentclass command.
\documentclass[sigconf]{acmart}
\settopmatter{printacmref=false}
%%
%% \BibTeX command to typeset BibTeX logo in the docs
\AtBeginDocument{%
  \providecommand\BibTeX{{%
    \normalfont B\kern-0.5em{\scshape i\kern-0.25em b}\kern-0.8em\TeX}}}

% TODO
%% Rights management information.  This information is sent to you
%% when you complete the rights form.  These commands have SAMPLE
%% values in them; it is your responsibility as an author to replace
%% the commands and values with those provided to you when you
%% complete the rights form.
\setcopyright{iw3c2w3}
\copyrightyear{\the\year{}}
\acmYear{\the\year{}}

\begin{document}

\title[Prompting Behavior in LLM Interactions]{Exploring User Prompting Behavior in LLM Interactions}

\author{Maximilian Slapnik}
%\orcid{1234-5678-9012}
\email{Maximilian.Slapnik@campus.lmu.de}
\affiliation{%
  \institution{LMU Munich}
  \city{Munich}
  \country{Germany}
}

%%%%%%%%%%%%%%%%%%%%%%%%%%%%%%%%%%%%%%% ABSTRACT

\begin{abstract}
  \sloppy
  Artificial Intelligence (AI) plays an increasingly important role in the daily lives of
  millions of people.
  Large Language Models (LLMs) are one of the most prominent implementations of AI that are used
  not only by experts, but equally by ordinary users as well.
  LLMs can respond to any textual input (prompts) with human-like answers, leveraging the
  training data that was used to implement the model.
  Even though prompting LLMs seems very straightforward, the question arises if it is possible
  to streamline the interactions with said models in order to optimize outputs.
  We investigate user behavior in interactions with LLMs based on a randomized trial of 100
  samples that are publicly available on the platforms ShareGPT and Midjourney.
  The goal of this analysis is the discovery of recurring patterns as well as the
  evaluation of human tendencies and biases when interacting with AI models in
  order to understand prevalent behaviors and explore optimization opportunities.
\end{abstract}

%%%%%%%%%%%%%%%%%%%%%%%%%%%%%%%%%%%%%%% CCS CONCEPTS

%% code below is generated by the tool at http://dl.acm.org/ccs.cfm.

\begin{CCSXML}
  <ccs2012>
  <concept>
  <concept_id>10003120.10003123</concept_id>
  <concept_desc>Human-centered computing~Interaction design</concept_desc>
  <concept_significance>500</concept_significance>
  </concept>
  <concept>
  <concept_id>10002951.10003317</concept_id>
  <concept_desc>Information systems~Information retrieval</concept_desc>
  <concept_significance>300</concept_significance>
  </concept>
  <concept>
  <concept_id>10010147.10010178.10010179</concept_id>
  <concept_desc>Computing methodologies~Natural language processing</concept_desc>
  <concept_significance>300</concept_significance>
  </concept>
  </ccs2012>
\end{CCSXML}

\ccsdesc[500]{Human-centered computing~Interaction design}
\ccsdesc[300]{Infor-mation systems~Information retrieval}
\ccsdesc[300]{Computing methodo-logies~Natural language processing}

%%%%%%%%%%%%%%%%%%%%%%%%%%%%%%%%%%%%%%% KEYWORDS

\keywords{Large Language Models, user behavior, prompting, interaction patterns}

%%%%%%%%%%%%%%%%%%%%%%%%%%%%%%%%%%%%%%% TITLE

\maketitle


%%%%%%%%%%%%%%%%%%%%%%%%%%%%%%%%%%%%%%% CHAPTERS

%%%%%%%%%%%%%%%%%%%%%%%%%%%%%%% INTRODUCTION %%%%%%%%%%%%%%%%%%%%%%%%%%%%%%%

%%%%%%%%%%%%%%%%%%%%%%%%%%%%%%%%% Introduction %%%%%%%%%%%%%%%%%%%%%%%%%%%%%%%%%

\section{Introduction}
\label{sec:introduction}
%% refer to https://intra.ece.ucr.edu/~rlake/Whitesides_writing_res_paper.pdf for tips on introduction?
%% INTRO - TOPIC

%Why is the work important?
\sloppy % use sloppy to improve linebreaks - longer words do not overflow
Artificial Intelligence (AI) -based tools continually gain prominence as regularly leveraged tools in the
daily lives of millions of people.
In addition to typical AI applications such as recommendation systems or autonomous agents, generative
models are notably increasing in popularity as well.
One of the most widely used implementations of generative models are Large Language Models (LLMs),
such as OpenAI's ChatGPT, which is the most prominent at the moment, for example. % TODO cite
Such models mainly come in the form of text generating chatbots that can answer seemingly any question
a user might pose.
Nevertheless, it is challenging to optimize the output of the model, since it can vary depending
on the user input. % TODO cite?
Any request to an LLM, whether it is in the form of a task or a question, is commonly referred to as
\("\)prompting\("\)the model.
Due to the vast application possibilities and promising future developments of LLMs,
an exploration of user prompting behavior in interactions with these models is of particular interest.

%- The objectives of the work.
In this paper, we are going to explain the workings of LLMs and prompting, describe related research
in the realm of user - LLM exchange, and perform our own investigation of user behavior in these
interactions.
This investigation will provide an improved understanding of existing challenges users face when
dealing with such models, as well as highlight optimization potential in order to enhance generated
output.

%- Background:
    % Who else has done what? How? What have we done previously?
Plenty of research has been conducted in the field of user interactions with LLMs.
% TODO list related work + describe briefly - fill in after completion of related work section?


%- Guidance to the reader:
% What should the reader watch for in the paper?
% What are the interesting high points? What strategy did we use?
Since the main part of this work will contain an analysis of real world examples, the reader can
expect to gain a better understanding of user prompting behavior.
To obtain these insights, we will leverage input data mainly obtained from the website
ShareGPT, % TODO cite
which stores voluntarily shared conversations of users with the ChatGPT model. % TODO cite

%- Summary/conclusion:
% What should the reader expect as conclusion?
In the concluding section of this paper we will summarize our findings and explain how and in which way
we can recognize findings from related research in our own data samples.

%%%%%%%%%%%%%%%%%%%%%%%%%%%%%%% BACKGROUND %%%%%%%%%%%%%%%%%%%%%%%%%%%%%%%

%%%%%%%%%%%%%%%%%%%%%%%%%%%%%%%%% Background %%%%%%%%%%%%%%%%%%%%%%%%%%%%%%%%%
\section{Background}
\label{sec:background}

%%%%%%%%%%%%%%%%%%%%%%%%%%%%%%%%% LLMs %%%%%%%%%%%%%%%%%%%%%%%%%%%%%%%%%%%%
\subsection{Large Language Models (LLMs)}
\label{subsec:large-language-models-(llms)}

There are many applications of generative AI, but the most widely used today are Large Language Models.
Among these LLMs, the most widely adopted is ChatGPT~\cite{openai_chatgpt_2023}, which is being
developed by OpenAI\@.
The model is currently publicly accessible free of charge. % TODO cite

% What is an LLM? What does it look like?
As is the case with ChatGPT, most LLMs designed for end users are implemented as a chatbot.
They typically consist of an interface made up of an input field for the user to type in arbitrary text, as well as
an output area that displays generated responses of the model.

% How does an LLM / ChatGPT work?
ChatGPT is based on a transformer architecture % TODO cite
and therefore leverages neural networks in order to be able to dynamically generate responses according
to user inputs.

% Why are LLMs important for our work? How do they come into play?


%%%%%%%%%%%%%%%%%%%%%%%%%%%%%%% USER STUDY %%%%%%%%%%%%%%%%%%%%%%%%%%%%%%%

%%%%%%%%%%%%%%%%%%%%%%%%%%%%%%%%% STUDY ON USAGE PATTERNS OF LLM USERS %%%%%%%%%%%%%%%%%%%%%%%%%%%%%%%%%
\section{Study on Usage Patterns of LLM Users}
\label{sec:study-on-usage-patterns-of-llm-users}

% Ideas for language prompt coding categories:
%- Prompt type: could include questions, statements, commands, or specific task-based prompts
%- User intent: could include seeking information, asking for advice, requesting clarification,
% expressing opinions, or making suggestions
%- Response quality: might include accurate and helpful responses, incomplete or ambiguous
% responses, irrelevant responses, or responses requiring further clarification
%- User satisfaction: could include satisfied, dissatisfied, neutral, confused, or impressed
%- Engagement level: might include active conversation, single-turn interactions, probing for
% more information, or exploratory questioning
%- Task success: could include successful completion, partial success, or failure
%- Response length
%- Ethical considerations: might include potential biases in the generated responses, adherence
% to ethical guidelines, or fairness and inclusivity in the LLM's behavior

% % Ideas for image prompt coding categories:
% Prompt type: capture the types of inputs users provide to the image generation models
% User intent: Extend the user intent category to encompass image-related intents--could
% include users requesting image generation, describing desired visual characteristics, or
% seeking specific visual outcomes.
% Response quality: evaluate the quality of the generated images--consider aspects such as visual
% fidelity, realism, relevance to the user's request, or adherence to given guidelines.
% User satisfaction: not possible? maybe on discord (but hard to discern)
% Engagement level: might involve exploring how users refine their requests, provide feedback on
% generated images, or iteratively interact with the model to achieve their desired visual
% outcomes. (hard to discern, same as above)
% Task success: might include successful image generation, partially successful
% results, or cases where the model failed to meet the user's expectations.
% Ethical considerations: could involve examining issues such as biases in generated images,
% ethical implications of content creation, privacy concerns related to user-provided images, or
% responsible use of image generation models.

% analysis: do we see differences between text and image prompts? which is more successful /
% accurate in general? do we observe sentiment differences?

% Analyze distribution of zero-, one-, and few-shot prompts
% analyze frequency of prompt reformulations in order to improve results
%

\subsection{Research Objective}
\label{subsec:research-objective}
% Intro and Research Objective of the study goal, the methodology, and the individual
% steps that will be taken
The main outlined goal of our research to gain a fundamental understanding of user
interaction in conversation with Large Language Models.
This analysis includes identifying common patterns and strategies in those interactions.
Previous research indicates, that users regularly face challenges and difficulties, especially
when trying to formulate effective prompts.
Through analysis of qualified data samples we aim to identify and understand these
challenges,
as well as investigate the impact and effect of user behavior on the effectiveness of LLM responses.
Given the various kinds of available generative models, we want to examine differences in
prompting behavior according to model type as well.

Since related insights suggest that reformulating search queries is a popular strategy to improve
results, we want to investigate if users apply this strategy in LLM conversations as well. % TODO
% check if we really did this
Furthermore, we want to assess extent to which users show awareness of effective prompt
formulation strategies, such as few-shot learning, and whether they rely on appropriate language
that is machine and not human directed, thus showing comprehension of the fundamental difference
between talking to a machine versus a human.

\subsection{Research Method: ShareGPT and Midjourney}
\label{subsec:research-method:-sharegpt-and-midjourney}
% Information on the ShareGPT & Midjourney platforms, their user base, suitability for the
% study, and which data we are going to use
In order to obtain credible insights, we complement existing findings with real-world data.
Our study analyzes data samples from two different sources, which involve different types of
LLMs.
First, we examine user interactions with ChatGPT, a generative NLP model.
These conversations were obtained from the website ShareGPT\cite{sharegpt_sharegpt_2023}.
ShareGPT is an open platform, that allows its community to publicly share conversations they have
had with the ChatGPT model.
As of today, ShareGPT has accumulated nearly 300.000 saved user conversations.
What makes ShareGPT particularly suitable for our use-case is the fact that the entire shared conversation
can then be viewed by the observer as if they had personally conducted the interaction.
This feature allows us to gain a deeper understanding of the conversation dynamics and
outcome.
By randomly choosing 50 conversations from ShareGPT, we obtain a representative sample of average
user interactions.
\newline

Midjourney in contrast, is a platform that focuses on AI-based image generation instead of text.
Users can interact with the model through Discord and submit individual prompts. % TODO cite Discord
In order to generate an illustration, a user has to enter a descriptive prompt, similar to
ChatGPT\@.
The description is usually comprised of everything that should appear in the picture, but may also
include the desired mood, drawing style, or composition of the image being generated.
Notably, the Midjourney Bot does not understand grammar, sentence structure, or specific words like
humans. % TODO cite
The developers actively encourage using fewer, but more precise and impactful words when prompting
the model.
For example, they suggest using \("\)gigantic\("\) instead of \("\)big\("\) in order to achieve
better results. %TODO cite
This recommendation stems from the fact that fewer words in a prompt intensify the influence each
individual word has on the final outcome.
However, % Be specific: anything you leave out will be randomized
Finally, Midjourney allows image inputs in addition to textual prompts.
Users may provide an image including instructions about things to modify, add, remove, or remodel.



% TODO section or subsection?


\section{Study Results}
\label{sec:study-results}

\subsection{}
% Listing of the results of the study, potentially segregated into categories that can be defined
% in advance

% TODO make an own subsection on difficulties of ordinary users in understanding how LLMs interpret
% natural language? in this subsec, we could focus on the example also displayed on midjourneys
% website: " If you ask for a party with “no cake,” your image will probably include a cake"


%%%%%%%%%%%%%%%%%%%%%%%%%%%%%%% DISCUSSION %%%%%%%%%%%%%%%%%%%%%%%%%%%%%%%

%%%%%%%%%%%%%%%%%%%%%%%%%%%%%%%%% DISCUSSION %%%%%%%%%%%%%%%%%%%%%%%%%%%%%%%%%
\section{Discussion}
\label{sec:discussion}
In this section, we reason about the observed behavior from the two data sources and try to offer
informed assumptions on potential causes.
% 1. Why do users interact with LLMs the way they do? Reasoning and informed assumptions on the
% causes of observed behavior
% 2. Prompt Improvement Possibilities Proposition of ways to enhance prompts as well as associated
% results based on findings from related researc

% first ChatGPT, then Midjourney analysis, then comparison?

\subsection{Data Synthesis: Commonalities, Differences, and Possible Explanations}
\label{subsec:data-synthesis:-commonalities-differences-and-possible-explanations}

\subsubsection{ChatGPT Behavior}
In regard to the type of prompt, we observed mainly task-based and question-related queries from
users.
This leads us to imagine that some users treat LLMs such as ChatGPT increasingly as their
personal online assistant when it comes to executing various tasks the model might be able to solve.
Leveraging AI bots as assistants of the future is a use case that gains popularity and is
increasingly researched~\cite{eshghie_chatgpt_2023}.
The second most prevalent type of prompt were questions.
This observations strengthens the possibility, that NLP models might replace current search engines
in the future.
This belief is further confirmed by observations from other researches who already see this
emerging trend in user behavior~\cite{van_bulck_what_2023}.
Regarding prompt intent, looking for information was the most popular use case.
Seemingly, users see ChatGPT as a reliable source of knowledge and trust its abilities.
This assumption can be dangerous however, as it is well-known that every LLM is probabilistic and
only as good as its training data.
Results should therefore always be verified.
Based on the fact that a lot of users also relied on ChatGPT for suggestions, we assume that it
is gladly used as a means to get ahead when you hit roadblocks, or need support in endeavours
that require creativity.
As we already touched on above, we did not have any requests for opinions of the model.
We initially thought that opinion related requests on difficult, morally complex, or
controversial topics would be more popular, since such behavior could be observed multiple times
when users tried to test the limits of the model, or for example when researchers tested political
biases~\cite{rozado_political_2023}.
Regarding the prompt setting, we made an observation similar to Brown et al.~\cite{
    brown_language_2020}.
Users do not leverage effective prompting techniques such as a few shot approach.
Instead, they relied heavily (90\%) on zero shot prompts.
We attribute this behavior to missing awareness of users about optimized techniques, and
therefore recommend that LLM providers actively inform users, e.g.\ by
providing examples, or releasing guidelines.
Our observations when looking at user engagement makes us think that current LLMs are already
quite accurate: almost half of the users ended the conversation after the initial answer of
the LLM (single turn).
This is further reinforced by 66\% of prompts that were not refined.
We therefore suspect, that users were actually content with the results most of the time.
However, we have to mention the possibility that the initial answer was so far off,
that users could have simply stopped trying after the first attempt.
Overall, we reckon that users prefer to leverage LLMs for rather simple tasks.
It is difficult to say, whether users do not trust LLMs enough to throw complex questions at
them yet, or if it is in the general nature of online search requests that the majority of
them are rather simple.
Few users refine their prompts (66\% do not).
Related work however suggests that refining queries has led to improved results when using search
engines.
LLMs can be improved by refinements too, since models are “primed” by all previous prompts in
the interaction.
So even if a model does not initially do what you want it to, it might make
sense to give it more information or context, and try again.
Due to missing sentiments and feedback of users in the sample interactions, we cannot reliably say
if users were simply always content because they did not refine, or if they did not know that
continuing the interaction with the model could have led to better results.
Finally, we have generally observed a majority of formal, generally polite, and acceptable language.
We reason that the use of such language ties in with our first observation, and users see
the bot as a personal online companion, that is therefore well treated.

\subsubsection{Midjourney Behavior}
For Midjourney, we identified a majority of language inputs.
We assume, this tendency exists because users prefer to create something new instead
of reworking existing images.
We imagine AI could eventually revolutionise the image editing market as well, but apparently is
not quite there yet.
This is probably due to missing accuracy when trying to make only very small
adjustments.
Regarding prompt length, the official Midjourney docs state: \("\)The Midjourney Bot works best with
simple, short sentences that describe what you want to see\("\)~\cite{midjourney_documentation_2023}.
Long sentences should be avoided, but we have seen users ignore this advice multiple times.
There sometimes seems to be a lack of understanding that more information is not
always better for quality of model outputs.
This observation fits another that has been made before:
Users struggle to formulate precise, effective, and therefore also short, concise prompts.
We suggested better learning materials and guidance already as a measure to address this issue.
Regarding composition of the Midjourney prompts, we suspect that this struggle is also the reason
for the rare occurrence of a keywords-only prompts.
Users seem to prefer natural language sentences or at least a mix of sentences and keywords
instead of purely "encoding" their wants.
To help, developers could offer reformulation engines, that only extract keywords (or create them)
from input sentences.
Since almost no images were regenerated as a reaction to the initial result of a prompt
we had classified as complex, we conclude that Midjourney is able to handle difficult queries well.
This assumption is confirmed by the fact that users did generally not refine their queries a lot
after the initial result was shown by the engine.
It was difficult to judge formality of language of users, but in general users tended to rely on
formal language, which was probably also influenced by the keyword-focused nature of the prompts.
We attribute the fact that most queries were formulated clearly to this circumstance as well,
since keyword focused prompts are usually shorter and less complicated than NLP prompts.

\subsubsection{Commonalities and Differences between ChatGPT and Midjourney}
A clear distinction between ChatGPT and Midjourney prompts lies in the nature of the model and
their designs.
Whereas ChatGPT is trained on full sentence queries, Midjourney is keyword focused.
However, we have remarked already that Midjourney users tend to at least partially integrate
sentence structures in their prompts as well, which may be due to the fact that it seems more
natural and is easier for users due to existing habits.
Both models were prompted with language on the more formal side.
It is possible, that this perception is skewed though, as prompts with informal language are
either not shared in the case of ChatGPT, or are directly filtered before execution in the case
of Midjourney.
It is worth mentioning that we were able to observe a previously mentioned general misconception
that researchers had identified in conversations of users with natural language LLMs already.
Some users rely on negation when providing instructions and misunderstand that it will not
prevent the model from producing the unwanted.
Our data samples from Midjourney contain one example where the user provides an image of the
fictional character Voldemort from the book and movie saga \("\)Harry Potter\("\).
They then explicitly ask the model to
generate an image of "voldemort dying without a nose".
Since the sole existence of the word "nose" in the prompt primes the model towards including said
object, all four result images indeed contain visualizations of the fictional character voldemort
passing--but with a nose.
Remarkably, there might have been a higher chance that the generated Voldemort does not possess a nose in the final image,
if the user did not explicitly mention the word, as
Voldemort does not actually have a human-like nose in the books and movies.
The prompt therefore achieved the opposite effect.


%%%%%%%%%%%%%%%%%%%%%%%%%%%%%%% CONCLUSION %%%%%%%%%%%%%%%%%%%%%%%%%%%%%%%

%%%%%%%%%%%%%%%%%%%%%%%%%%%%%%%%% CONCLUSION %%%%%%%%%%%%%%%%%%%%%%%%%%%%%%%%%
\section{Conclusion}
\label{sec:conclusion}

Throughout this paper we have explored user behavior in interactions with LLMs across multiple
dimensions.
The goal was to find prevalent human tendencies, understand existing habits, and identify recurrent
patterns.

To do so, we first explained the ever-growing importance of the subject, given the increasing
use of generative AI across all domains.
We then laid out fundamental concepts and explained existing findings from related research.
These concepts and findings built the foundation for a synthesis of our own real-world
data analysis with existing research in order to verify findings.
To obtain a comprehensive overview, we explained different kinds of LLMs and their
characteristics, highlighting ChatGPT and Midjourney in particular.
An extensive study of data samples that encompassed a categorization of each individual
entry according to eight predefined categories helped us to understand the current state of
human - LLM interactions.
We could observe that many users are still subject to biases and misunderstandings that make
effective prompting difficult.
During the discussion of our results, we reasoned why certain user dynamics exists, and which
actions LLM providers and developers as well as users themselves could undertake in order to
address prevalent issues.

Deeper exploration of this topic with a larger data sample is needed in order to
verify findings and gain a deeper understanding of existing interaction dynamics and is subject to
further research.
Overall, it is to be acknowledged that user prompting behavior is one of the most relevant
research topics in human centered computing and will gain even more importance in the near future.

%%%%%%%%%%%%%%%%%%%%%%%%%%%%%%%%%%%%%%% BIBLIOGRAPHY
\newpage
\bibliographystyle{ACM-Reference-Format}
\bibliography{bibliography}

\end{document}
\endinput
