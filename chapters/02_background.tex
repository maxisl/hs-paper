%%%%%%%%%%%%%%%%%%%%%%%%%%%%%%%%% Background %%%%%%%%%%%%%%%%%%%%%%%%%%%%%%%%%
\section{Background}
\label{sec:background}

%%%%%%%%%%%%%%%%%%%%%%%%%%%%%%%%% LLMs %%%%%%%%%%%%%%%%%%%%%%%%%%%%%%%%%%%%
\subsection{Large Language Models (LLMs)}
\label{subsec:large-language-models-(llms)}

There are many applications of generative AI, but the most widely used today are Large Language Models.
Among these LLMs, the most widely adopted is ChatGPT~\cite{openai_chatgpt_2023}, which is being developed by OpenAI
The model is currently publicly accessible free of charge. % TODO cite

% What is an LLM? What does it look like?
As is the case with ChatGPT, most LLMs that are designed for end users appear as a chatbot, and
consist of an interface made up of an input field for the user to type in arbitrary text, as well as
an output area where generated responses of the model get displayed.

% How does an LLM work?
ChatGPT is based on a transformer architecture % TODO cite
and therefore leverages neural networks in order to be able to dynamically generate responses according
to user inputs.

% Why are LLMs important for our work? How do they come into play?
