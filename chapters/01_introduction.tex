%%%%%%%%%%%%%%%%%%%%%%%%%%%%%%%%% Introduction %%%%%%%%%%%%%%%%%%%%%%%%%%%%%%%%%

\chapter{Introduction}
\label{ch:introduction}
%% refer to https://intra.ece.ucr.edu/~rlake/Whitesides_writing_res_paper.pdf for tips on introduction?
%% INTRO - TOPIC
Artificial Intelligence (AI) -based tools continually gain prominence as regularly leveraged tools
in the daily lives of millions of people.
In addition to typical applications such as recommendation systems or autonomous agents,
generative models are notably gaining popularity.
One of the most widely used implementations are Large Language Models (LLMs).
Since their main application is text generation in the form of an AI chatbot that can answer seemingly
any question a user might pose, an investigation of such interactions is particularly of interest.

%One of the main challenges is
%% connectivity is a given
%Today,  is ever prevalent, which is reflected in the fact that
%
%
%% LLMs
%
%% Prompting
Giving an LLM a task or asking it a question is commonly referred to as \("\)prompting\("\) the model.

%
%% User interaction / input + what will be done?
%

%In general, the Introduction should have these elements:
%- The objectives of the work.
%- The justification for these objectives:
    %Why is the work important?
%- Background:
    %Who else has done what? How? What have we done previously?
%- Guidance to the reader:
    %What should the reader watch for in the paper?
    %What are the interesting high points? What strategy did we use?
%- Summary/conclusion:
    %What should the reader expect as conclusion?
    %In advanced versions of the outline, you should also include all the sections
    %that will go in the Experimental section (at the level of paragraph subheadings)
    %and indicate what information will go in the Microfilm section.