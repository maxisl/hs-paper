%%%%%%%%%%%%%%%%%%%%%%%%%%%%%%%%% Introduction %%%%%%%%%%%%%%%%%%%%%%%%%%%%%%%%%
\section{Introduction}
\label{sec:introduction}
%% refer to https://intra.ece.ucr.edu/~rlake/Whitesides_writing_res_paper.pdf for tips on introduction?
%% INTRO - TOPIC

%Why is the work important?
\sloppy % use sloppy to improve linebreaks - longer words do not overflow
Artificial Intelligence (AI) -based tools continually gain prominence as regularly leveraged tools in the
daily lives of millions of people.
Today, the significance of this technology is reflected in the current AI market size that is
estimated to be \$142 billion USD, and forecasted to increase more than tenfold by 2030~\cite{statista_artificial_2023}.
% TODO include some stats, such as daily ChatGPT users?
In addition to typical AI applications such as recommendation systems or autonomous agents, generative
models are notably increasing in popularity as well, making it one of the central research topics
in the field.
One of the most widely used implementations of generative models are Large Language Models (LLMs),
the most popular example at the moment being OpenAI's ChatGPT~\cite{openai_chatgpt_2023}.
Adoption rates of generative AI among professionals are increasing rapidly, and are already at
around 30\%~\cite{statista_us_2022}.

Large Language Models are mainly implemented in the form of text generating chatbots that can
answer seemingly any question a user might pose.
Although no expert knowledge is required to formulate a prompt and interact with an LLM-based bot, it is challenging to optimize the output, since it can vary depending on the structure, wording, and composition of the user input. %TODO cite?
Any form of model input, whether it is in the form of a task or a question, is commonly
referred to as \("\)prompting\("\) the model.
Vast application possibilities and promising future developments of LLMs make an exploration of
user prompting behavior in interactions with such models particularly interesting.

%- The objectives of the work.
In this paper, we are going to explain the fundamentals and workings of LLMs and prompting,
describe related research in the realm of user - LLM exchange, and perform our own investigation of
user behavior in these interactions.
This investigation has the objective of facilitating comprehension of existing challenges users
face when dealing with Large Language Models.
Furthermore, readers will gain a better understanding of the design of effective prompts that
enhance
generated model output.

%- Background:
% Who else has done what? How? What have we done previously?
Plenty of research has been conducted in the field of user interactions with LLMs, mainly in
regard to query reformulation strategies, studies of common user errors when prompting,
different prompt formulation strategies, and LLM limitations in general.
We will highlight relevant work in a dedicated related work section. % TODO link section?
% TODO list related work + describe briefly - fill in after completion of related work section?

%- Guidance to the reader:
% What should the reader watch for in the paper?
% What are the interesting high points? What strategy did we use?
Since the main part of this paper will be complimented by an analysis of real world examples, the
reader can expect to gain a better understanding of actual user prompting behavior.
To obtain these insights, we will leverage input data mainly gathered from the website
ShareGPT~\cite{sharegpt_sharegpt_2023}, % TODO cite
which enables users to store conversations they have had with the ChatGPT model for later retrieval
or sharing them publicly.

%- Summary/conclusion:
% What should the reader expect as conclusion?
% TODO add points as paper progresses

% TODO add links (\ref) to sections
The paper is organized as follows.
The introduction provides a general overview of the paper, introducing the topic and setting the context for all
subsequent parts.
The following related work section is divided into two subsections.
We first (2.1) focus on Large Language Models (LLMs) and cover general information about their
workings, training data, text generation capabilities, real-world usage, and current limitations.
The second subsection (2.2) explores user interaction with LLMs, explains the concept of
prompting and highlights various use cases as well as related research.

Section 3 begins with an introduction to the study, outlining our research objective, and
describing the methodology and individual steps.
It then focuses on the research method used, as well as the ShareGPT and Midjourney platforms,
including details about their user base and suitability for the study, since we use
samples from both as input data.

Subsequently, we present the study results.
To do so, we first list the findings obtained from the study, organized into predefined categories.
% TODO check if true
We then analyze observable trends in user behavior and data patterns and visualize them.

The discussion section starts with a synthesis of our observations.
We then go into more detail about the reasons why users interact with LLMs the way they do,
offering reasoning and informed assumptions.
Additionally, we explore possibilities for prompt improvement based on findings from related
research.

In the outlook section, we provide a perspective on future developments, divided into an
introduction of the concept of Auto-GPT as a possible future iteration of prompting, as well as an
overview of prompt engineering as a newly emerging discipline in the technology sector.

The final section of this paper contains a summary of the findings and associated discussions and
explains how and in which way or form we could recognize findings from related research in our
own input samples.