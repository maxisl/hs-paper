%%%%%%%%%%%%%%%%%%%%%%%%%%%%%%%%% STUDY ON USAGE PATTERNS OF LLM USERS %%%%%%%%%%%%%%%%%%%%%%%%%%%%%%%%%
\section{Study on Usage Patterns of LLM Users}
\label{sec:study-on-usage-patterns-of-llm-users}

asd

% Ideas for language prompt coding categories:
%- Prompt type: could include questions, statements, commands, or specific task-based prompts
%- User intent: could include seeking information, asking for advice, requesting clarification,
% expressing opinions, or making suggestions
%- Response quality: might include accurate and helpful responses, incomplete or ambiguous
% responses, irrelevant responses, or responses requiring further clarification
%- User satisfaction: could include satisfied, dissatisfied, neutral, confused, or impressed
%- Engagement level: might include active conversation, single-turn interactions, probing for
% more information, or exploratory questioning
%- Task success: could include successful completion, partial success, or failure
%- Response length
%- Ethical considerations: might include potential biases in the generated responses, adherence
% to ethical guidelines, or fairness and inclusivity in the LLM's behavior

% % Ideas for image prompt coding categories:
% Prompt type: capture the types of inputs users provide to the image generation models
% User intent: Extend the user intent category to encompass image-related intents--could
% include users requesting image generation, describing desired visual characteristics, or
% seeking specific visual outcomes.
% Response quality: evaluate the quality of the generated images--consider aspects such as visual
% fidelity, realism, relevance to the user's request, or adherence to given guidelines.
% User satisfaction: not possible? maybe on discord (but hard to discern)
% Engagement level: might involve exploring how users refine their requests, provide feedback on
% generated images, or iteratively interact with the model to achieve their desired visual
% outcomes. (hard to discern, same as above)
% Task success: might include successful image generation, partially successful
% results, or cases where the model failed to meet the user's expectations.
% Ethical considerations: could involve examining issues such as biases in generated images,
% ethical implications of content creation, privacy concerns related to user-provided images, or
% responsible use of image generation models.

% concluding: do we see differences between text and image prompts? which is more successful /
% accurate in general? do we observe sentiment differences?

% Analyze distribution of zero-, one-, and few-shot prompts
% analyze frequency of prompt reformulations in order to improve results
%