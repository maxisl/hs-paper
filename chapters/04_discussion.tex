%%%%%%%%%%%%%%%%%%%%%%%%%%%%%%%%% DISCUSSION %%%%%%%%%%%%%%%%%%%%%%%%%%%%%%%%%
\section{Discussion}
\label{sec:discussion}
% 1. Why do users interact with LLMs the way they do? Reasoning and informed assumptions on the
% causes of observed behavior
% 2. Prompt Improvement Possibilities Proposition of ways to enhance prompts as well as associated
% results based on findings from related researc

% first ChatGPT, then Midjourney analysis, then comparison?

\subsection{Data Synthesis: Commonalities, Differences, and Possible Explanations}
\label{subsec:data-synthesis:-commonalities-differences-and-possible-explanations}
In this section, we reason about the observed behavior from the two data sources and try to offer
informed assumptions on potential causes.

\subsubsection{ChatGPT Behavior}
In regard to the type of prompt, we observed mainly task-based and question-related queries from
users.
This leads us to imagine that some users treat LLMs such as ChatGPT increasingly as their
personal online assistant when it comes to executing various tasks the model might be able to solve.
Leveraging AI bots as assistants of the future is a use case that gains popularity, and that is also
increasingly researched~\cite{eshghie_chatgpt_2023}.
The second most prevalent type of prompt were questions.
We have already mentioned, that NLP models might replace current search engines in the future.
This belief is further confirmed by observations from other researches who already see this
emerging trend in user behavior~\cite{van_bulck_what_2023}.
Regarding prompt intent, looking for information was the most popular use case.
Seemingly, users see ChatGPT as a reliable source of knowledge and trust its abilities.
This assumption can be dangerous however, as it is well-known that every LLM is probabilistic and
only as good as its training data.
Results should therefore always be verified.


\subsubsection{Midjourney Behavior}

\subsubsection{Commonalities and Differences between ChatGPT and Midjourney}
