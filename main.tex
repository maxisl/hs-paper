\usepackage{enumitem}

%% The first command in your LaTeX source must be the \documentclass command.
\documentclass[sigconf]{acmart}
\settopmatter{printacmref=false}
%%
%% \BibTeX command to typeset BibTeX logo in the docs
\AtBeginDocument{%
  \providecommand\BibTeX{{%
    \normalfont B\kern-0.5em{\scshape i\kern-0.25em b}\kern-0.8em\TeX}}}

%% Rights management information.  This information is sent to you
%% when you complete the rights form.  These commands have SAMPLE
%% values in them; it is your responsibility as an author to replace
%% the commands and values with those provided to you when you
%% complete the rights form.
\setcopyright{iw3c2w3}
\copyrightyear{\the\year{}}
\acmYear{\the\year{}}


%%
%% end of the preamble, start of the body of the document source.
\begin{document}

%%
%% The "title" command has an optional parameter,
%% allowing the author to define a "short title" to be used in page headers.
\title[Prompting Behavior in LLM Interactions]{Exploring User Prompting Behavior in LLM Interactions}

%%
%% The "author" command and its associated commands are used to define
%% the authors and their affiliations.
%% Of note is the shared affiliation of the first two authors, and the
%% "authornote" and "authornotemark" commands
%% used to denote shared contribution to the research.
\author{Maximilian Slapnik}
%\orcid{1234-5678-9012}
\email{Maximilian.Slapnik@campus.lmu.de}
\affiliation{%
  \institution{LMU Munich}
  \city{Munich}
  \country{Germany}
}


%%
%% By default, the full list of authors will be used in the page
%% headers. Often, this list is too long, and will overlap
%% other information printed in the page headers. This command allows
%% the author to define a more concise list
%% of authors' names for this purpose.
%\renewcommand{\shortauthors}{Trovato and Tobin, et al.}

%%
%% The abstract is a short summary of the work to be presented in the
%% article.
\begin{abstract}
  Artificial Intelligence (AI) plays an increasingly important role in the daily lives of millions of people.
  Large Language Models (LLMs) are the most prominent implementation of AI that is used not only by experts, but equally by ordinary users as well.
  LLMs can respond to any textual input (prompts) with human-like answers, leveraging the training data that was used to implement the model.
  Even though prompting LLMs seems very straightforward, the question arises if it is possible to streamline the interactions with said models in order to optimize outputs.
  We explore the behavior of a randomized trial of 100 interactions of users with LLMs that are publicly available on ShareGPT\@.
  The goal of this investigation is the discovery of recurring patterns in behavior and the evaluation of human tendencies as well as biases of users when interacting with AI models in order to understand current behaviors and propose optimization opportunities.
\end{abstract}

%%%%%%%%%%%%%%%%%%%%%%%%%%%%%%%%%%%%%%% CCS CONCEPTS

%%
%% The code below is generated by the tool at http://dl.acm.org/ccs.cfm.
%% Please copy and paste the code instead of the example below.
%%
\begin{CCSXML}
  <ccs2012>
  <concept>
  <concept_id>10003120.10003123</concept_id>
  <concept_desc>Human-centered computing~Interaction design</concept_desc>
  <concept_significance>500</concept_significance>
  </concept>
  <concept>
  <concept_id>10002951.10003317</concept_id>
  <concept_desc>Information systems~Information retrieval</concept_desc>
  <concept_significance>300</concept_significance>
  </concept>
  <concept>
  <concept_id>10010147.10010178.10010179</concept_id>
  <concept_desc>Computing methodologies~Natural language processing</concept_desc>
  <concept_significance>300</concept_significance>
  </concept>
  </ccs2012>
\end{CCSXML}

\ccsdesc[500]{Human-centered computing~Interaction design}
\ccsdesc[300]{Information systems~Information retrieval}
\ccsdesc[300]{Computing methodologies~Natural language processing}

%%%%%%%%%%%%%%%%%%%%%%%%%%%%%%%%%%%%%%% KEYWORDS

\keywords{Large Language Models, user behavior, prompting, interaction patterns}

%%%%%%%%%%%%%%%%%%%%%%%%%%%%%%%%%%%%%%% TITLE

\maketitle


% 1. Introduction
\section{Introduction}

% 2. Background and Related Work
\section{Background and Related Work}
  \subsection{Large Language Models (LLMs)}
$\circ$ General information on LLMs, such as their workings, training data, text generation, real world usage, and current limitations
  \subsection{User Interaction with LLMs}
$\circ$ Explanation of Prompting \newline
$\circ$ Description of LLM use cases and related work, primarily paying attention to ordinary frequent users as we lay a particular focus on daily, and not only expert usage

% 3. Study on Usage Patterns of LLM Users
\section{Study on Usage Patterns of LLM Users}
  \subsection{Intro and Research Objective}
$\circ$ Overview of the study goal, the methodology, and the individual steps that will be taken
  \subsection{Research Method: ShareGPT}
$\circ$ Information on the ShareGPT platform, its user base, its suitability for the study, and which data we are going to use
  \subsection{Study Results}
    \subsubsection{Findings}
$\circ$ Listing of the results of the study, potentially segregated into categories that can be defined in advance
    \subsubsection{Observable Trends}
$\circ$ Objective analysis of results with a particular focus on observable trends in user behavior and data patterns (including visualizations such as charts)

% 4. Discussion
\section{Discussion}
  \subsection{Observed Behaviour (Synthesis)}
$\circ$ Subjective evaluation of findings
    \subsubsection{Why do users interact with LLMs the way they do?}
$\circ$ Reasoning and informed assumptions on the causes of observed behavior
    \subsubsection{Prompt Improvement Possibilities}

$\circ$ Proposition of ways to enhance prompts as well as associated results based on findings from related research
  \subsection{Outlook and Future Developments}
    \subsubsection{Auto-GPT}
$\circ$ Introduction to future developments in the realm of LLM interaction, such as AI-based agents which may execute prompts autonomously in the future
    \subsubsection{Prompt Engineering}
$\circ$ Focus on the newly emerging discipline of prompt engineering which is a direct result of the increased significance of LLMs and required competencies for successful interaction

% 5. Conclusion
\section{Conclusion}

%%%%%%%%%%%%%%%%%%%%%%%%%%%%%%% Introduction %%%%%%%%%%%%%%%%%%%%%%%%%%%%%%%

%%%%%%%%%%%%%%%%%%%%%%%%%%%%%%%%% Introduction %%%%%%%%%%%%%%%%%%%%%%%%%%%%%%%%%
\section{Introduction}
\label{sec:introduction}
%% refer to https://intra.ece.ucr.edu/~rlake/Whitesides_writing_res_paper.pdf for tips on introduction?
%% INTRO - TOPIC

%Why is the work important?
\sloppy % use sloppy to improve linebreaks - longer words do not overflow
Artificial Intelligence (AI) -based tools continually gain prominence as regularly leveraged tools in the
daily lives of millions of people.
Today, the significance of this technology is reflected in the current AI market size that is
estimated to be \$142 billion USD, and forecasted to increase more than tenfold by 2030~\cite{statista_artificial_2023}.
% TODO include some stats, such as daily ChatGPT users?
In addition to typical AI applications such as recommendation systems or autonomous agents, generative
models are notably increasing in popularity as well, making it one of the central research topics
in the field.
One of the most widely used implementations of generative models are Large Language Models (LLMs),
the most popular example at the moment being OpenAI's ChatGPT~\cite{openai_chatgpt_2023}.
Adoption rates of generative AI applications among professionals are increasing rapidly, and are
already at
around 30\%~\cite{statista_us_2022}.

Large Language Models are mainly implemented in the form of text generating chatbots that can
answer seemingly any question a user might pose.
Although no expert knowledge is required to formulate a prompt and interact with an LLM-based bot, it is challenging to optimize the output, since it varies depending on the structure, wording,
and composition of the input. %TODO cite?
Any form of model input, whether it is in the form of a task or a question, is commonly
referred to as \("\)prompting\("\) the model.
Due to the vast application possibilities and promising future developments of LLMs, exploration of
user prompting behavior in interactions with such models is of particular interest.
Plenty of research has been conducted in the field of user interactions with LLMs already,
mainly in regard to query reformulation strategies, studies of common user errors when prompting,
different prompt composition strategies, and general LLM limitations.

%- The objectives of the work.
In this paper, we are going to explain the fundamentals and workings of LLMs and prompting,
describe related research in the realm of user - LLM exchange, and perform our own investigation of
user behavior in such interactions.
This investigation has the objective of facilitating comprehension of existing challenges users
face when dealing with Large Language Models.
Furthermore, readers will gain a better understanding of the design of effective prompts that
enhance model output.

%- Guidance to the reader:
% What should the reader watch for in the paper?
% What are the interesting high points? What strategy did we use?
Since the main part of this paper will be complemented by an analysis of real-world examples, the
reader can expect to develop an enhanced comprehension of actual user prompting behavior.
To obtain these insights, we will leverage input data mainly gathered from the website
ShareGPT~\cite{sharegpt_sharegpt_2023},
which enables users to store conversations they have had with the ChatGPT model for later retrieval
or sharing them publicly.

% TODO add links (\ref) to sections
The paper is organized as follows.
This introduction is succeeded by a related work section that sets the context for all subsequent
parts by first focusing on Large Language Models (LLMs) and covering general information about
their workings, training data, text generation capabilities, real-world usage, and current limitations.
We then explore user interactions with LLMs, explain the concept of prompting, and highlight
various use cases as well as related research.

The next section introduces the study by outlining the research objective and
describing the methodology and individual steps that will be taken.
It then focuses on the research method we use, as well as the ShareGPT and Midjourney platforms,
which provide the input data for the study.

Subsequently, we present our findings.
To do so, we first list the study results, organized into predefined categories.
% TODO check if true
We then analyze observable trends in user behavior and data patterns.

The following discussion section starts with a synthesis of our observations.
We then go into more detail about the reasons why users interact with LLMs the way they do,
offering reasoning and informed assumptions.
Additionally, we explore possibilities for prompt improvements based on findings from related
research.

In the outlook section, we provide a perspective on future developments, divided into an
introduction of the concept of Auto-GPT as a possible future iteration of prompting, as well as an
overview of prompt engineering as a newly emerging discipline in the technology sector.

The final section of this paper offers a consolidation of the findings and associated discussions,
as well as a summary of how we could recognize findings from related research in our own input
samples.

%%%%%%%%%%%%%%%%%%%%%%%%%%%%%%%%%%%%%%% BIBLIOGRAPHY
\bibliographystyle{ACM-Reference-Format}
\bibliography{bibliography}

\end{document}
\endinput
%%
%% End of file `sample-sigconf.tex'.
