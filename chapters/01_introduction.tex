%%%%%%%%%%%%%%%%%%%%%%%%%%%%%%%%% Introduction %%%%%%%%%%%%%%%%%%%%%%%%%%%%%%%%%

\section{Introduction}
\label{sec:introduction}
%% refer to https://intra.ece.ucr.edu/~rlake/Whitesides_writing_res_paper.pdf for tips on introduction?
%% INTRO - TOPIC

%Why is the work important?
\sloppy % use sloppy to improve linebreaks - longer words do not overflow
Artificial Intelligence (AI) -based tools continually gain prominence as regularly leveraged tools in the
daily lives of millions of people.
% TODO include some stats, such as daily ChatGPT users?
In addition to typical AI applications such as recommendation systems or autonomous agents, generative
models are notably increasing in popularity as well.
One of the most widely used implementations of generative models are Large Language Models (LLMs),
the most popular example at the moment being OpenAI's ChatGPT~\cite{openai_chatgpt_2023}.
These models mainly come in the form of text generating chatbots that can answer seemingly any question
a user might pose.
Nevertheless, it is challenging to optimize the output of the model, since it can vary depending
on the user input. % TODO cite?
Any form of such input to an LLM, whether it is in the form of a task or a question, is commonly
referred to as \("\)prompting\("\)the model.
Due to the vast application possibilities and promising future developments of LLMs,
an exploration of user prompting behavior in interactions with these models is of particular interest.

%- The objectives of the work.
In this paper, we are going to explain the fundamentals and workings of LLMs and prompting,
describe related research in the realm of user - LLM exchange, and perform our own investigation of
user behavior in these interactions.
This investigation will provide an improved understanding of existing challenges users face when
dealing with such models, as well as highlight optimization potential in order to enhance generated
output.

%- Background:
    % Who else has done what? How? What have we done previously?
Plenty of research has been conducted in the field of user interactions with LLMs.
% TODO list related work + describe briefly - fill in after completion of related work section?

%- Guidance to the reader:
% What should the reader watch for in the paper?
% What are the interesting high points? What strategy did we use?
Since the main part of this work will contain an analysis of real world examples, the reader can
expect to gain a better understanding of actual user prompting behavior.
To obtain these insights, we will leverage input data mainly gathered from the website
ShareGPT, % TODO cite
which enables users to store conversations they have had with the ChatGPT model and share them
with others.

%- Summary/conclusion:
% What should the reader expect as conclusion?
% TODO add points as paper progresses
In the concluding section of this paper we will summarize our findings and explain how and in which way
we can recognize findings from related research in our own data samples.
% TODO add info about general structure / sections of paper: "The paper is organized as follows..."