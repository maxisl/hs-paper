%%%%%%%%%%%%%%%%%%%%%%%%%%%%%%%%% Background %%%%%%%%%%%%%%%%%%%%%%%%%%%%%%%%%
\section{Background}
\label{sec:background}

%%%%%%%%%%%%%%%%%%%%%%%%%%%%%%%%% LLMs %%%%%%%%%%%%%%%%%%%%%%%%%%%%%%%%%%%%
\subsection{Large Language Models (LLMs)}
\label{subsec:large-language-models-(llms)}

There are many applications of generative AI, but the most widely used today are Large Language Models.
Among these LLMs, the most widely adopted is ChatGPT~\cite{openai_chatgpt_2023}, which is being
developed by OpenAI\@.
The model is currently publicly accessible free of charge.

% What is an LLM? What does it look like?
As is the case with ChatGPT, most LLMs designed for end users are implemented as a chatbot.
They typically consist of an interface made up of an input field for the user to type in arbitrary text, as well as
an output area that displays generated responses of the model.

% How does an LLM / ChatGPT work?
Large Language Models originated after the development of the original transformer architecture,
which is a deep learning approach first introduced by researchers in 2017~\cite{vaswani_attention_2017}.
Based on this architecture, the Generative Pre-Training (GPT)~\cite{radford_improving_2018} approach
was developed for text-based models in particular, which is the foundation for today's most popular
LLMs, such as ChatGPT\@.
Since our research is focused on user interaction with text-based models, we will not highlight other
application and development areas of LLMs in particular.


% Why are LLMs important for our work? How do they come into play?
